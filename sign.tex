\documentclass[10pt,-letter paper]{article}
\usepackage[left=1in, right=0.75in, top=1in, bottom=0.75in]{geometry}
\usepackage{graphicx} % Required for inserting images
\usepackage{siunitx}
\usepackage{setspace}
\usepackage{gensymb}
\usepackage{xcolor}
\usepackage{caption}
%\usepackage{subcaption}
\doublespacing
\singlespacing
\usepackage[none]{hyphenat}
\usepackage{amssymb}
\usepackage{relsize}
\usepackage[cmex10]{amsmath}
\usepackage{mathtools}
\usepackage{amsmath}
\usepackage{commath}
\usepackage{amsthm}
\interdisplaylinepenalty=2500
%\savesymbol{iint}
\usepackage{txfonts}
%\restoresymbol{TXF}{iint}
\usepackage{wasysym}
\usepackage{amsthm}
\usepackage{mathrsfs}
\usepackage{txfonts}
\let\vec\mathbf{}
\usepackage{stfloats}
\usepackage{float}
\usepackage{cite}
\usepackage{cases}
\usepackage{subfig}
%\usepackage{xtab}
\usepackage{longtable}
\usepackage{multirow}
%\usepackage{algorithm}
\usepackage{amssymb}
%\usepackage{algpseudocode}
\usepackage{enumitem}
\usepackage{mathtools}
%\usepackage{eenrc}
%\usepackage[framemethod=tikz]{mdframed}
\usepackage{listings}
%\usepackage{listings}
\usepackage[latin1]{inputenc}
%%\usepackage{color}{   
%%\usepackage{lscape}
\usepackage{textcomp}
\usepackage{titling}
\usepackage{hyperref}
%\usepackage{fulbigskip}   
\usepackage{tikz}
\usepackage{graphicx}
\lstset{
  frame=single,
    breaklines=true
    }
    \let\vec\mathbf{}
    \usepackage{enumitem}
    \usepackage{graphicx}
    \usepackage{siunitx}
    \let\vec\mathbf{}
    \usepackage{enumitem}
    \usepackage{graphicx}
    \usepackage{enumitem}
    \usepackage{tfrupee}
    \usepackage{amsmath}
    \usepackage{amssymb}
    \usepackage{mwe} % for blindtext and example-image-a in example
    \usepackage{wrapfig}
    \graphicspath{{figs/}}
    \providecommand{\cbrak}[1]{\ensuremath{\left\{#1\right\}}}
    \providecommand{\brak}[1]{\ensuremath{\left(#1\right)}}
    \newcommand{\sgn}{\mathop{\mathrm{sgn}}}
    \providecommand{\abs}[1]{\left\vert#1\right\vert}
    \providecommand{\res}[1]{\Res\displaylimits_{#1}} 
    
    \providecommand{\norm}[1]{\left\lVert#1\right\rVert}
    %\providecommand{\norm}[1]{\lVert#1\rVert}
    \providecommand{\mtx}[1]{\mathbf{#1}}
    \providecommand{\mean}[1]{E\left[ #1 \right]}
    \providecommand{\fourier}{\overset{\mathcal{F}}{ \rightleftharpoons}}
    %\providecommand{\hilbert}{\overset{\mathcal{H}}{ \rightleftharpoons}}
    \providecommand{\system}{\overset{\mathcal{H}}{ \longleftrightarrow}}
     %\newcommand{\solution}[2]{\textbf{Solution:}{#1}}
     %\newcommand{\solution}{\noindent \textbf{Solution: }}
     \newcommand{\cosec}{\,\text{cosec}\,}
     \providecommand{\dec}[2]{\ensuremath{\overset{#1}{\underset{#2}{\gtrless}}}}
     \newcommand{\myvec}[1]{\ensuremath{\begin{pmatrix}#1\end{pmatrix}}}
     \newcommand{\myaugvec}[2]{\ensuremath{\begin{amatrix}{#1}#2\end{amatrix}}}
     \newcommand{\mydet}[1]{\ensuremath{\begin{vmatrix}#1\end{vmatrix}}}
     \title{MATHEMATICS}
     \author{SECTION A}
     \date{\today}
     \begin{document}

     \maketitle

     \begin{enumerate}
     \section{Vectors}
    \item Find the angle between the line $\overrightarrow{r}=\brak{2\hat{i}-\hat{j}+3\hat{k}}+\lambda\brak{3\hat{i}-\hat{j}+2\hat{k}}$ and the plane $\overrightarrow{r}.\brak{\hat{i}+\hat{j}+\hat{k}}=3$.

     \section{Matrices}

     \item Find the value of $\brak{x-y}$ from the matrix equation $2\myvec{x & 5 \\ 7 & y-3} + \myvec{-3 & -4 \\ 1 & 2}= \myvec{7 & 6 \\ 15 & 14}$

     \item Using elementary row transformations, find the inverse of the matrix $\myvec{3 & 0 & -1\\
	2 & 3 & 0\\
	0 & 4 & 1\\}$.

 \item Using matrices, solve the following system of linear equations:
	\begin{align*}
		2x+3y+10z=4\\
		4x-6y+5z=1\\
		6x+9y-20z=2
	\end{align*}

      \item Find the equation of the plane passing through $\brak{-1,3,2}$ and perpendicular to the planes $x+2y+3z=5$ and $3x+3y+z=0$.
      
     \section{Probability}
     
     \item If $A$ and $B$ are independent events with $P\brak{A}=\frac{3}{7}$ and $P\brak{B}=\frac{2}{5}$, then find $P\brak{A'\bigcap B'}$.

     \item A card from a pack of $52$ playing cards is lost. From the remaining cards of the pack, two cards are drawn at random (without replacement) and both are found to be spades. Find the probability of the lost card being a spade.
     
 
     \section{Differentiation}

     \item Find the differential equation representing the family of curves $y=-A \cos 3x+B \sin 3x$.

    \item Find the differential of the function $\cos^{-1}\brak{\sin 2x}$ w.r.t. $x$.

    \item Solve the following differential equation :
\begin{align*}          
	 \brak{y + 3x^{2}}\frac{dx}{dy}=x
\end{align*}


    \item If $y = \brak{\sin{x}^x}+\sin^{-1}\brak{\sqrt{1-x^2}}$, then find $\frac{dy}{dx}$
    
     \section{Integration}
    \item Find:
    \begin{align*}
		\int\frac{x-1}{\brak{x-2}\brak{x-3}}dx
\end{align*}

\item Find:
	\begin{align*}
		\int {e^x\brak{\frac{2+\sin 2x}{2 \cos^2 x}}} dx
	\end{align*}

\item Evaluate: 
	\begin{align*}
		\int_{1}^{5} \brak{\vert{x-1}\vert+\vert{x-2}\vert+\vert{x-4}\vert} dx
	\end{align*}
 
 \item Find:
	\begin{align*}
		\int\cos{2x}\cos{4x}\cos{6x}~dx
	\end{align*}
     
     \section{Geometry}
     
     \item Prove that the radius of the right circular cylinder of greatest curved surface area which can be inscribed in a given cone is half of that of the cone.
     
     \section{Function}
     \item Find the interval in which the function $f$ given by $f\brak{x}=\sin2x+\cos2x,0\leq{x}\leq{\pi}$ is strictly decreasing.

     \section{Optimization}
		\item A company manufactures two types of novelty souvenirs made of plywood. Souvenirs of type $A$ require $5$ minutes each for cutting and $10$ minutes each for assembling. Souvenirs of type $B$ require $8$ minutes each for cutting and $8$ minutes each for assembling. There are $3$ hours $20$ minutes available for cutting and $4$ hours for assembling. The profit for type $A$ souvenirs is \rupee~$100$ each and for type $B$ souvenirs, profit is \rupee~$120$ each. How many souvenirs of each type should the company manufacture in order to maximise the profit ? Formulate the problem as a LPP and then solve it graphically.		
\end{enumerate}
     \end{document}
